\documentclass{article}
\usepackage[latin1]{inputenc}
\usepackage[a4paper,left=2cm,right=2cm,top=2cm,bottom=2cm]{geometry}
\usepackage[T1]{fontenc}
\usepackage[french]{babel}
\usepackage{amsmath}
\usepackage{amsfonts}
\usepackage{dsfont}
\usepackage{graphicx}
\usepackage{caption}
\usepackage{listings}

\setlength{\parindent}{0pt}
\setlength{\parskip}{1ex plus 0.5ex minus 0.2ex}
\newcommand{\hsp}{\hspace{20pt}}
\newcommand{\HRule}{\rule{\linewidth}{0.5mm}}
\newcommand*{\logeq}{\ratio\Leftrightarrow}

\title{Projet IA - Probl\`eme de la patrouille - Approche EVAP}
\author{Timoth\'e Rios - Nicolas Venot}
\date{mars 2021}

\begin{document}
\maketitle
\newpage
\tableofcontents
\newpage
\setlength{\parindent}{0pt}
\section*{Introduction}
\paragraph{}Ce projet a pour objectif de mod\'eliser aussi pr\'ecis\'ement que possible le probl\`eme de patrouille en utilisant l'approche d'\'evaporation des ph\'eromones.
Cette approche, inspir\'ee par l'\'etude des colonies d'insectes sociaux, a pour particularit\'e d'attribuer \`a chaque patch un taux de 'chemical' sp\'ecifique qui diminue avec le temps.
Les agents ont alors pour fonction de maximiser autant que possible le taux de chemical des patchs autour d'eux, d'o\`u la n\'ecessit\'e de patrouiller et la pertinence de ce mod\`ele pour r\'esoudre ce probl\`eme.
\section{Fonctionnement g\'en\'eral}
    \subsection{Les patchs}
\paragraph{}Les patchs se voient attribuer un bool\'een indiquant si ceux-ci sont des murs ou non (aussi distingu\'e par une couleur bleu qui leur est sp\'ecifique), un param\`etre chemical charg\'e d'indiquer le niveau de ph\'eromones de chaque cases ainsi qu'un param\`etre ini d\'estin\'e \`a \^etre utilis\'e \`a des fins d'observation statistiques.
Les patchs se voit attribuer une couleur \`a chaque tick, celle-ci varie selon leur niveau de chemical :
plus celui-ci est proche de 1 plus le patch appara\^it vert, lorsqu'il se rapproche de 0 la case redevient noire. Le niveau de chemical est multipli\'e par 1 - p \`a chaque tick (p modulable) si le patch n'est pas un mur afin de simuler l'\'evaporation des ph\'eromones.
      
    \subsection{Les agents}
\paragraph{}Les agents sont charg\'es de maintenir le taux de chemical de tous les patchs le plus proche de 1 possible.
Ils fonctionnent de la mani\`ere suivante : \`a chaque tick, l'agent dresse la liste des patch ayant le niveau de chemical le plus faible parmi ses 4 voisins (uniquement sur la m\^eme ligne ou colonne). Une fois la liste cr\'e\'ee, l'agent va effectuer un mouvement ayant un pourcentage de chance d\'efini par le param\`etre 'go-ahead' de privil\'egier la case en face de lui si celle-ci fait partie de la liste, sinon l'agent fait simplement le mouvement vers une autre case de la liste en se tournant vers elle au pr\'ealable.

Apr\`es avoir atteint un nouveau patch, l'agent indique \`a celui-ci son nouveau niveau de chemical ( \'egal \`a 1, la case devient verte) et reset son param\`etre ini \`a 0. Les agents ignorent les patchs correspondant \`a des murs car ceux-ci se voit attribuer une valeur fixe de chemical par d\'efaut de 2, il ne peuvent donc jamais correspondre au minimum de la liste, ce qui emp\^eche l'agent d'engager un d\'eplacement vers un mur.
\section{Algorithme}
    \subsection{}
\section{\'Etude des param\`etres}
    \subsection{}
\section{Conclusion}
    \subsection{}

\appendix
\newpage
\section{R\'ecapitulatif des types et fonctions}


\end{document}