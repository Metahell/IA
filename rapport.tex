\documentclass{article}
\usepackage[latin1]{inputenc}
\usepackage[a4paper,left=2cm,right=2cm,top=2cm,bottom=2cm]{geometry}
\usepackage[T1]{fontenc}
\usepackage[french]{babel}
\usepackage{amsmath}
\usepackage{amsfonts}
\usepackage{dsfont}
\usepackage{graphicx}
\usepackage{caption}
\usepackage{listings}

\setlength{\parindent}{0pt}
\setlength{\parskip}{1ex plus 0.5ex minus 0.2ex}
\newcommand{\hsp}{\hspace{20pt}}
\newcommand{\HRule}{\rule{\linewidth}{0.5mm}}
\newcommand*{\logeq}{\ratio\Leftrightarrow}

\title{Projet IA - Problème de la patrouille - Approche EVAP}
\author{Timoth\'e Rios - Nicolas Venot}
\date{mars 2021}

\begin{document}
\maketitle
\newpage
\tableofcontents
\newpage
\setlength{\parindent}{0pt}
\section*{Introduction}
\paragraph{Ce projet a pour objectif de modéliser aussi précisément que possible le problème de patrouille en utilisant l'approche d'évaporation des phéromones.
Cette approche, inspirée par l'étude des colonies d'insectes sociaux, a pour particularité d'attribuer à chaque patch un taux de 'chemical' spécifique qui diminue avec le temps.
Les agents ont alors pour fonction de maximiser autant que possible le taux de chemical des patchs autour d'eux, d'où la nécessité de patrouiller et la pertinence de ce modèle pour résoudre ce problème.}
\section{1 Fonctionnement général}
    \subsection{1.1 Les patchs}

      
    \subsection{1.2 Les agents}
        
\section{2 Algorithme}
    \subsection{2.1}
\section{3 Conclusion}
    \subsection{3.1}

\appendix
\newpage
\section{R\'ecapitulatif des types et fonctions}


    \end{document}