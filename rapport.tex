\documentclass{article}
\usepackage[a4paper,left=2cm,right=2cm,top=2cm,bottom=2cm]{geometry}
\usepackage[utf8]{inputenc}  
\usepackage[T1]{fontenc}
\usepackage[french]{babel}
\usepackage{amsmath}
\usepackage{amsfonts}
\usepackage{dsfont}
\usepackage{graphicx}
\usepackage{caption}
\usepackage{listings}

\setlength{\parindent}{0pt}
\setlength{\parskip}{1ex plus 0.5ex minus 0.2ex}
\newcommand{\hsp}{\hspace{20pt}}
\newcommand{\HRule}{\rule{\linewidth}{0.5mm}}
\newcommand*{\logeq}{\ratio\Leftrightarrow}

\title{Projet IA - Problème de la patrouille - Approche EVAP}
\author{Timothé Rios - Nicolas Venot}
\date{mars 2021}

\begin{document}
\maketitle
\newpage
\tableofcontents
\newpage
\setlength{\parindent}{0pt}
\section*{Introduction}
\paragraph{}Ce projet a pour objectif de modéliser aussi précisément que possible le problème de patrouille 
en utilisant l'approche d'évaporation des phéromones.
Cette approche, inspirée par l'étude des colonies d'insectes sociaux, a pour 
particularité d'attribuer à chaque zone géographique un taux de phéromones spécifique qui diminue avec le temps.
Les agents ont alors pour fonction de maximiser autant que possible le taux de phéromones 
des zones autour d'eux, d'où la nécessité de patrouiller et la pertinence de ce modèle pour 
résoudre ce problème.

\section{Principe général}

\paragraph{} Dans ce modèle, le temps est discrétisé et l'environnement est modélisé par une grille de cases. Chaque agent
avance d'une case par pas de temps, sur la case à gauche, à droite, devant ou derrière lui qui a la plus petite quantité de phéromones. Chaque fois qu'un agent arrive sur
une case, il dépose une quantité $P_{max}$ de phéromones sur celle-ci. 

À chaque pas de temps, toutes les cases perdent une certaine quantité de phéromones : si elles ont $P_t$ phéromones
à l'instant $t$, elles ont $P_{t+1} = P_t * (1-p)$ à l'instant $t+1$, avec $p$ compris entre $0$ et $1$ exclus.

Une catégorie de cases ne peut pas recevoir de phéromones et ne peut pas être traversée par les agents. Cette catégorie permet
de représenter des murs ou des obstacles dans ce modèle. Évidemment, les agents en contact avec des cases de type mur
ne vont pas prendre en compte ces dernières dans leurs déplacements.

Enfin, si plusieurs cases atteignables par un agent au prochain pas de temps ont toutes deux le niveau de phéromones minimum 
parmi les voisins de cet agent, il choisit au hasard parmi celles-ci. Cependant, si une de ces cases est la case qui lui fait 
face, il a la probabilité $p_avant$ de garder sa trajectoire. Ceci permet d'éviter des mouvements trop erratiques.

\section{Implémentation}
    \subsection{Les cases}
\paragraph{}Chaque se voit attribuer un booléen indiquant si celle-ci est un mur ou non (aussi distingué par 
une couleur bleue qui leur est spécifique), un paramètre phéromone chargé d'indiquer le niveau de phéromones 
de chaque case ainsi qu'un paramètre ini déstiné à être utilisé à des fins d'observation statistiques.
Les cases se voit attribuer une couleur à chaque tick, celle-ci varie selon leur niveau de phéromones :
plus celui-ci est proche de 1 plus la case apparaît verte, lorsqu'il se rapproche de 0 la case redevient noire. 
Le niveau de phéromones est multiplié par 1 - p à chaque tick (p modulable) si la case n'est pas un mur afin 
de simuler l'évaporation des phéromones.
      
    \subsection{Les agents}
\paragraph{}Les agents sont chargés de maintenir le taux de phéromones de tous les cases le plus proche de 1 possible.
Ils fonctionnent de la manière suivante : à chaque tick, l'agent dresse la liste des case ayant le niveau 
de phéromones le plus faible parmi ses 4 voisins (uniquement sur la même ligne ou colonne). Une fois la liste
 créée, l'agent va effectuer un mouvement ayant un pourcentage de chance défini par le paramètre 'go-ahead' 
 de privilégier la case en face de lui si celle-ci fait partie de la liste, sinon l'agent fait simplement le 
 mouvement vers une autre case de la liste en se tournant vers elle au préalable.

Après avoir atteint un nouveau case, l'agent indique à celui-ci son nouveau niveau de phéromones ( égal à 1, 
la case devient verte) et remet son paramètre ini à 0. Les agents ignorent les cases correspondant à des murs 
car ceux-ci se voit attribuer une valeur fixe de phéromones par défaut de 2, il ne peuvent donc jamais correspondre 
au minimum de la liste, ce qui empêche l'agent d'engager un déplacement vers un mur.
\section{Algorithme}
    \subsection{}
\section{étude des paramètres}
    \subsection{}
\section{Conclusion}
    \subsection{}

\appendix
\newpage
\section{Récapitulatif des types et fonctions}


\end{document}